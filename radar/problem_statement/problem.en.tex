\problemname{Radar}
TODO story

There are $N$ points on a number line, and you want to find them by making a small number of queries that ask for the distance to the second closest point to some provided coordinate $x$.

That is, if the points have distances $d_1, d_2, \ldots, d_n$ to $x$, we sort them in increasing order and answer with $d_2$.

All coordinates are unique and between $1$ and $10^{18}$.

Depending on the number of queries used your program will receive different amounts of points.

\section*{Interaction}
Your program will interact with the judge by reading from standard in and writing to standard out.
First, the judge will output a line containing the number of points $N$ ($3 \le N \le 100$) and the number of the test group $T$ ($1 \le T \le 3$).

Then, your program can make queries of the following forms:

\begin{itemize}
  \item \texttt{? x} ($-10^{18} \le x \le 2\cdot 10^{18}$): find the distance to the second closest point to $x$.
    The judge will respond with two space-separated numbers $r$, $m$,
    where $r$ is the distance to the second closest point, and $m$ is the number of points at that distance (either $1$ or $2$).
  \item \texttt{! $a_1$ $a_2$ ... $a_n$}: guess the positions of the points ($1 \le a_i \le 10^{18}$).
    $a_1, a_2, \ldots, a_n$ may be given in any order.
    No additional queries should be made after this.
\end{itemize}

After each query, you should make sure to flush standard output before reading the judge's response, or your solution may end up graded as Time Limit Exceeded.
In Python this happens automatically, and in C++ it happens when using \verb@cout << endl@ to print a newline.
In C you can use \verb@fflush(stdout);@ after \texttt{printf()}.

If you make an invalid query, your program may receive \emph{either} Wrong Answer or Run Time Error.

To facilitate testing of your solution, we provide a simple tool that you can download from the sidebar of the Kattis page for this problem.
See the comment at the top of the file for a description of how to use it.

\section*{Scoring}
Your solution will be tested on a set of test groups, each worth up to a number of points.
To get the points for a test group you need to solve all test cases in the test group.

\noindent
\begin{tabular}{| l | l | l |}
  \hline
  Group & Points       & Constraints \\ \hline
  $1$   & $40 \cdot r$ & One of the points is located at position $1$ \\ \hline
  $2$   & $40 \cdot r$ & All coordinates are even \\ \hline
  $3$   & $20 \cdot r$ & No additional constraints \\ \hline
\end{tabular}

For each test group, let $Q$ be the maximum number of \texttt{?}-type queries used across any test case in that group.
Then for that group $r$ is defined according to the following table:

\noindent
\begin{tabular}{| l | l | l |}
  \hline
  Constraints                & $r$ \\ \hline
  $             Q > 15\,000$ & $0$ \\ \hline
  $ 15\,000 \ge Q > 5\,600$  & $0.4$  \\ \hline
  $  5\,600 \ge Q > 3\,500$  & $0.6$  \\ \hline
  $  3\,500 \ge Q > 2\,500$  & $0.8$  \\ \hline
  $  2\,500 \ge Q$           & $1$  \\ \hline
\end{tabular}
