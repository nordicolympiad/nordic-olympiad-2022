\problemname{Radar}
TODO story

There are $N$ points on a number line, and you want to find them by making a limited number of queries that ask for the distance to the second closest point to some provided coordinate $x$.

That is, if the points have distances $d_1, d_2, \ldots, d_n$ to $x$, we sort them in increasing order and answer with $d_2$.

All coordinates are unique and between $1$ and $10^{18}$.

\section*{Interaction}
Your program will interact with the judge by reading from standard in and writing to standard out.
First, the judge will output a line containing the number of points $N$ ($2 \le N \le 100$), and the maximum number of queries allowed $Q$.

Then, your program can make queries of the following forms:

\begin{itemize}
  \item \texttt{? x} ($1 \le x \le 10^{18}$): find the distance to the second closest point to $x$.
    The judge will respond with two space-separated numbers $r$, $m$,
    where $r$ is the distance to the second closest point, and $m$ is the number of points at that distance (either $1$ or $2$).
    You may make at most $Q$ queries of this type.
  \item \texttt{! $a_1$ $a_2$ ... $a_n$}: guess the positions of the points ($1 \le a_i \le 10^{18}$).
    $a_1, a_2, \ldots, a_n$ may be given in any order.
    No additional queries should be made after this.
\end{itemize}

After each query, you should make sure to flush standard output before reading the judge's response, or your solution may end up graded as Time Limit Exceeded.
In Python this happens automatically, and in C++ it happens when using \verb@cout << endl@ to print a newline.
In C you can use \verb@fflush(stdout);@ after \texttt{printf()}.

If you make more than $Q$ queries or make an invalid query, your program may receive \emph{either} Wrong Answer or Run Time Error.
To be guaranteed an unambiguous Wrong Answer verdict, you should terminate your program just before making the $Q+1$'th \texttt{?}-type query.

\section*{Scoring}
Your solution will be tested on a set of test groups, each worth a number of points.
To get the points for a test group you need to solve all test cases in the test group.

\noindent
\begin{tabular}{| l | l | l |}
  \hline
  Group & Points & Constraints \\ \hline
  $1$   & $20$   & $Q = 15\,000$, and one of the points will be located at position $1$ \\ \hline
  $2$   & $25$   & $Q = 5\,600$ \\ \hline
  $3$   & $30$   & $Q = 3\,500$ \\ \hline
  $4$   & $25$   & $Q = 2\,500$ \\ \hline
\end{tabular}
